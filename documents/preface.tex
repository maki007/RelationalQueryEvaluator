\chapter*{Introduction}
\addcontentsline{toc}{chapter}{Introduction}
Today's processors have multiple cores and it's single core performance is improving only very slow because of physical limitations. On the other hand number of cores is still increasing and we can assume that it will continue. That's why developing parallel software is crucial for improving overall performance.

Parallelization can be achieved manually or using some framework designed for it. For example there are frameworks like OpenMP or Intel TBB. Department of Software Engineering at Charles University in Prague developed it's own parallelization framework called Bobox\cite{bobox}.

Bobox is designed for parallel processing large amounts of data. It was specifically created to simplify and speed up parallel programming of certain class of problems - data computations based on non-linear pipeline. It was created to evaluate queries over relational data but it was succesfully used in implementation of XQuery and TriQuery engines.

Bobox contains from runtime environment and operators. Theses operators are called boxes and they are C++ implementation of data processing algorithm. Boxes use messages called envelopes to send processed data to each other. 

Bobox takes as input execution plan written in special language Bobolang\cite{bobolang}. It allows to define used boxes and simply connect then into directed acyclic graph. Bobolang specifies the structure of whole application and also the inner structure of each box. It can create highly optimized evaluation, which is capable of using the most of the hardware resources. The language has been tested in several applications and it turned out to be very powerful tool in data processing massive parallel application.

Most used databases are relational databases. They are based on the view of data organized in tables called relations. SQL\cite{database} ("Structured query language") is very important language based on relation databases. It is used for queering data, modifying content of tables and also the structure of tables. When we want to evaluate query we need to parse query text input into parse tree. This form will be transformed to relational algebra, which we call logical query plan. It will be optimized and physical plan is generated. Physical plan indicates not only operation performed, but also which order are they performed and what kind of algorithms are used for execution.

The main goal of this thesis is to implement part of SQL compiler. The input is query written in XML format in from of relational algebra. Program validates input, optimizes and transforms it to physical plan of given query. The output is execution plan for Bobox written in Bobolang.
