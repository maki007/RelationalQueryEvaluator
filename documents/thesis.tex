%%% Hlavní soubor. Zde se definují základní parametry a odkazuje se na ostatní části. %%%




%% Verze pro jednostranný tisk:
% Okraje: levý 40mm, pravý 25mm, horní a dolní 25mm
% (ale pozor, LaTeX si sám přidává 1in)
\documentclass[12pt,a4paper]{report}
\setlength\textwidth{145mm}
\setlength\textheight{247mm}
\setlength\oddsidemargin{15mm}
\setlength\evensidemargin{15mm}
\setlength\topmargin{0mm}
\setlength\headsep{0mm}
\setlength\headheight{0mm}
% \openright zařídí, aby následující text začínal na pravé straně knihy
\let\openright=\clearpage

\usepackage{setspace}
\onehalfspacing
%% Pokud tiskneme oboustranně:
% \documentclass[12pt,a4paper,twoside,openright]{report}
% \setlength\textwidth{145mm}
% \setlength\textheight{247mm}
% \setlength\oddsidemargin{15mm}
% \setlength\evensidemargin{0mm}
% \setlength\topmargin{0mm}
% \setlength\headsep{0mm}
% \setlength\headheight{0mm}
% \let\openright=\cleardoublepage

%% Použité kódování znaků: obvykle latin2, cp1250 nebo utf8:
\usepackage[utf8]{inputenc}

%% Ostatní balíčky
\usepackage{graphicx}
\usepackage{amsthm}
\usepackage{latexsym}
\usepackage{mathtools}

\usepackage{amssymb}
\usepackage{ifsym}
\usepackage{listings}
\newtheorem{mydef}{Definition}
%% Balíček hyperref, kterým jdou vyrábět klikací odkazy v PDF,
%% ale hlavně ho používáme k uložení metadat do PDF (včetně obsahu).
%% POZOR, nezapomeňte vyplnit jméno práce a autora.
\usepackage[unicode]{hyperref}   % Musí být za všemi ostatními balíčky



\hypersetup{pdftitle=Název práce}
\hypersetup{pdfauthor=Marcel Kikta}

%%% Drobné úpravy stylu

% Tato makra přesvědčují mírně ošklivým trikem LaTeX, aby hlavičky kapitol
% sázel příčetněji a nevynechával nad nimi spoustu místa. Směle ignorujte.
\makeatletter
\def\@makechapterhead#1{
  {\parindent \z@ \raggedright \normalfont
   \Huge\bfseries \thechapter. #1
   \par\nobreak
   \vskip 20\p@
}}
\def\@makeschapterhead#1{
  {\parindent \z@ \raggedright \normalfont
   \Huge\bfseries #1
   \par\nobreak
   \vskip 20\p@
}}
\makeatother



% Toto makro definuje kapitolu, která není očíslovaná, ale je uvedena v obsahu.
\def\chapwithtoc#1{
\chapter*{#1}
\addcontentsline{toc}{chapter}{#1}
}

\begin{document}

% Trochu volnější nastavení dělení slov, než je default.
\lefthyphenmin=2
\righthyphenmin=2

%%% Titulní strana práce

\pagestyle{empty}
\begin{center}

\large

Charles University in Prague

\medskip

Faculty of Mathematics and Physics

\vfill

{\bf\Large MASTER THESIS}

\vfill

\centerline{\mbox{\includegraphics[width=60mm]{img/logo.pdf}}}

\vfill
\vspace{5mm}

{\LARGE Marcel Kikta}

\vspace{15mm}

% Název práce přesně podle zadání
{\LARGE\bfseries Evaluating relational queries in pipeline-based environment}

\vfill

% Název katedry nebo ústavu, kde byla práce oficiálně zadána
% (dle Organizační struktury MFF UK)
Department of Software Engineering

\vfill

\begin{tabular}{rl}

Supervisor of the master thesis: & David Bednárek \\
\noalign{\vspace{2mm}}
Study programme: & Software systems \\
\noalign{\vspace{2mm}}
Specialization: & Software engineering \\
\end{tabular}

\vfill

% Zde doplňte rok
Prague 2014

\end{center}

\newpage

%%% Následuje vevázaný list -- kopie podepsaného "Zadání diplomové práce".
%%% Toto zadání NENÍ součástí elektronické verze práce, nescanovat.

%%% Na tomto místě mohou být napsána případná poděkování (vedoucímu práce,
%%% konzultantovi, tomu, kdo zapůjčil software, literaturu apod.)

\openright

\noindent
I would like to thank my parents for supporting me in my studies and my thesis supervisor David Bednárek for his advice and help with this thesis.

\newpage

%%% Strana s čestným prohlášením k diplomové práci

\vglue 0pt plus 1fill

\noindent
I declare that I carried out this master thesis independently, and only with the cited
sources, literature and other professional sources.

\medskip\noindent
I understand that my work relates to the rights and obligations under the Act No.
121/2000 Coll., the Copyright Act, as amended, in particular the fact that the Charles
University in Prague has the right to conclude a license agreement on the use of this
work as a school work pursuant to Section 60 paragraph 1 of the Copyright Act.

\vspace{10mm}

\hbox{\hbox to 0.5\hsize{%
In ........ date ............
\hss}\hbox to 0.5\hsize{%
signature of the author
\hss}}

\vspace{20mm}
\newpage

%%% Povinná informační strana diplomové práce

\vbox to 0.5\vsize{
\setlength\parindent{0mm}
\setlength\parskip{5mm}

Název práce:
Vyhodnocování relačních dotazů v proudově orientovaném prostředí
% přesně dle zadání

Autor:
Marcel Kikta

Katedra:  % Případně Ústav:
Katedra softwarového inženýrství
% dle Organizační struktury MFF UK

Vedoucí diplomové práce:
RNDr. David Bednárek, Ph.D.
% dle Organizační struktury MFF UK, případně plný název pracoviště mimo MFF UK

Abstrakt:
% abstrakt v rozsahu 80-200 slov; nejedná se však o opis zadání diplomové práce

Klíčová slova:
SQL, Prěkladač, Relační algebra, Optimalizator, Bobox

\vss}\nobreak\vbox to 0.49\vsize{
\setlength\parindent{0mm}
\setlength\parskip{5mm}

Title:
% přesný překlad názvu práce v angličtině

Author:
Marcel Kikta

Department:
Název katedry či ústavu, kde byla práce oficiálně zadána
% dle Organizační struktury MFF UK v angličtině

Supervisor:
RNDr. David Bednárek, Ph.D.
% dle Organizační struktury MFF UK, případně plný název pracoviště
% mimo MFF UK v angličtině

Abstract:
% abstrakt v rozsahu 80-200 slov v angličtině; nejedná se však o překlad
% zadání diplomové práce

Keywords:
SQL, Compiler, Relational algebra, optimizer, Bobox

\vss}

\newpage

%%% Strana s automaticky generovaným obsahem diplomové práce. U matematických
%%% prací je přípustné, aby seznam tabulek a zkratek, existují-li, byl umístěn
%%% na začátku práce, místo na jejím konci.

\openright
\pagestyle{plain}
\setcounter{page}{1}
\tableofcontents

%%% Jednotlivé kapitoly práce jsou pro přehlednost uloženy v samostatných souborech
\chapter*{Introduction}
\addcontentsline{toc}{chapter}{Introduction}
Today's processors have multiple cores and it's single core performance is improving only very slow because of physical limitations. On the other hand number of cores is still increasing and we can assume that it will continue. That's why developing parallel software is crucial for improving overall performance.

Parallelization can be achieved manually or using some framework designed for it. For example there are frameworks like OpenMP or Intel TBB. Department of Software Engineering at Charles University in Prague developed it's own parallelization framework called Bobox\cite{bobox}.

Bobox is designed for parallel processing large amounts of data. It was specifically created to simplify and speed up parallel programming of certain class of problems - data computations based on non-linear pipeline. It was created to evaluate queries over relational data but it was succesfully used in implementation of XQuery and TriQuery engines.

Bobox contains from runtime environment and operators. Theses operators are called boxes and they are C++ implementation of data processing algorithm. Boxes use messages called envelopes to send processed data to each other. 

Bobox takes as input execution plan written in special language Bobolang\cite{bobolang}. It allows to define used boxes and simply connect then into directed acyclic graph. Bobolang specifies the structure of whole application and also the inner structure of each box. It can create highly optimized evaluation, which is capable of using the most of the hardware resources. The language has been tested in several applications and it turned out to be very powerful tool in data processing massive parallel application.

Most used databases are relational databases. They are based on the view of data organized in tables called relations. SQL\cite{database} ("Structured query language") is very important language based on relation databases. It is used for queering data, modifying content of tables and also the structure of tables. When we want to evaluate query we need to parse query text input into parse tree. This form will be transformed to relational algebra, which we call logical query plan. It will be optimized and physical plan is generated. Physical plan indicates not only operation performed, but also which order are they performed and what kind of algorithms are used for execution.

The main goal of this thesis is to implement part of SQL compiler. The input is query written in XML format in from of relational algebra. Program validates input, optimizes and transforms it to physical plan of given query. The output is execution plan for Bobox written in Bobolang.

\chapter{Architecture}

\section{Bobox}

In the section we describe basic architecture of Bobox. Information source for this chapter is Doctoral thesis Parallel Processing of Data\cite{faltthesis}. 

Overall Bobox architecture is displayed in figure~\ref{fig:bobox}. Framework contains of Boxes. Box is basically a C++ class containing implementation of data processing algorithm or it can be set of connected boxes. Box can have arbitrary number of inputs and outputs. All boxes are connected to a directed acyclic graph.  

\begin{figure}[h!]
  \centering
    \includegraphics[width=1\textwidth]{bobox}

      \caption{Bobox architecture.}
          \label{fig:bobox}
\end{figure}

Data streams are implemented as data units called enveloped. Envelope structure is displayed in figure~\ref{fig:envelope}. It consists of sequence tuples, but internally data are stored by columns, that means envelope contains from sequence of columns and it's data is stored in separate list. So to read all attributes of the i-th tuple we have to access all column lists and read it's i-th element. There is special type of envelope having poisoned pill. It is send after all valid data indicating end of data stream. 

\begin{figure}[h!]
  \centering
    \includegraphics[width=1\textwidth]{envelope}

      \caption{Envelope structure.}
          \label{fig:envelope}
\end{figure}
There are two special boxes, which have to be in every execution plan:
\begin{itemize}


\item $init$ - first box in topological order and it indicates starting box of execution plan

\item $term$ - last box in topological order and indicates that plan has been completely evaluated

\end{itemize}

Evaluation starts with scheduling $init$ box, which sends poisoned pills to all of its output. All of it's output boxes will be scheduled. They can read data from hard drive or network, process it and sent it to other boxes for further processing. Other boxes usually receives data in envelopes in their inputs. Box $term$ waits for every it's input to receive poisoned pill and then evaluation ends.

\section{Bobolang}
In this section we describe syntax and semantics of Bololang language. We used paper Bobolang - a language for parallel streaming applications\cite{bobolang} as information source.

Bobolang is a formal description language for Bobox execution plan. Bobox environment provides implementation of basic operators (boxes). Bobolang let's programmer choose which boxes to used, what boxes to use, what type are passed and how the boxes are interconnected. Bobolang also provides possibility to create operators connecting existing ones. 

In following example we show a definition using other operators:

\begin{verbatim}
operator process (int)->(int,int,int)
{
    preproc(int)->(int,int) pre;
    post(int,int)->(int,int,int) post;
	
    input -> pre;
    pre -> post -> output;
}
\end{verbatim}

Code specifies that we are creating new operator called \verb|process|. It takes one stream of integers as input and outputs one stream of triplets integers. 

In the first part we declare sub operators, define type of input and output. For every declared sub operator we provide identifier. Second part specifies connection between declared operators. Code \verb|op1 -> op2| indicates that output of \verb|op1| is connected to input of operator \verb|op2|. In this case output type of \verb|op1| has equal to input type of \verb|op2|. Bobolang syntax also allows to create chains of operators like \verb|op1 -> op3| which has semantics like \verb|op1 -> op2| and \verb|op2 -> op3|. 

There are explicitly defined operators called \verb|input| and \verb|output|. They represents input and output of declared operator \verb|process|. The line \verb|input -> pre;| represents that input of the operator \verb|process| is connected to operator \verb|pre|.

Boblang also allows to declare operators with empty input or output. They have type \verb|()| that means it doesn't transfer any data. Only data allowed is to transfer poisoned pill. When box receives poisoned pill, it means that it should start working, Sending it means that it's work is done.

We can define whole execution plan using operator main with empty input and output. Example of whole Bobolang plan:

\begin{verbatim}
operator main()->()
{
    source()->(int) src;
    process(int)->(int,int,int) proc;
    sink(int,int,int)->() sink;

    input -> src -> proc -> sink -> output;
}
\end{verbatim}
 In figure~\ref{fig:exampleplan} we can seen structure of example execution plan. Operators \verb|init| and \verb|term| are added automatically. Operator \verb|init| sends poisoned pill to \verb|source|, which can read data from hard drive or network. These data are send to box \verb|process|. Operator sink stores data and sends poisoned pill to box \verb|term| and the computation ends.
\begin{figure}[h!]
  \centering

    \includegraphics[width=1\textwidth]{exampleplan}
    
      \caption{Example of execution plan.}
        \label{fig:exampleplan}
\end{figure}


\section{SQL compiler architecture}
In this section we describe planed SQL compiler. It's architecture is displayed in figure~\ref{fig:sqlarchitecture}. 
\begin{figure}[h!]
  \centering

    \includegraphics[width=1\textwidth]{sqlarchitecture}
    
      \caption{SQL compiler architecture.}
        \label{fig:sqlarchitecture}
\end{figure}

SQL query is written in text. This text is parsed into parse tree, which is transformed into logical query plan (Relational algebra). Relational algebra is then optimized and this form is used for generating physical query plan. Physical plan written in Bobolang is input for Bobox for execution. Physical plan is not enough, we need to also provide implementation of physical algorithms (Bobox operators).

Since SQL is a pretty complicated language, this thesis aim is only implementing optimization and transformation of logical plan into physical plan.



\chapter{Related work}
I the chapter we introduce some theory of relational algebra, it's optimizations and physical plan generation. This informations were used for tool implementation.
\section{Relational algebra}

In this chapter we introduce and describe relational algebra\cite{database}. We start with some basic definitions of relational model.

\begin{mydef}
$Relation$ is a two dimensional table.
\end{mydef}
\begin{mydef}
$Attribute$ is column of a table.
\end{mydef}
\begin{mydef}
$Schema$ name of the relations and a set of attributes. For example:~$Movie(id,name,lenght)$.
\end{mydef}
\begin{mydef}
$Tupple$ of a relation is a row other than header row.
\end{mydef}


An algebra in general consist of  operators and atomic operands. For example in arithmetic algebra variables like $x$ or constant like 15 and operators are addition multiplication, subtraction and division.
We can build expression by applying operators on operands or other expressions. Example of an expression in arithmetic algebra is $(15+x)*x$.

Relational algebra  has atomic operands:
\begin{itemize}
\item Variables, that are relations.
\item Constants, that are finite relations. 
\end{itemize}

In classical relational algebra all operates and expression results are set. All this operations can be applied also to bags.
Relation algebra operators are:
\begin{itemize}
\item Set operations - union, difference, intersection.
\item Removing operators - selection, which removes rows and projection that eliminates columns from given relation.
\item Operations that combine two relation, all kinds of joins.
\item Renaming operations, that doesn't change tuples of the relation bud changes schema.
\end{itemize}
Expressions in relational algebra are called $queries$.
\subsection{Classical relational algebra operators}
\subsubsection{Set operations on relations}

Sets operations are:
\begin{itemize}
\item Union $R\cap S$ is a set of tuples that are in $R$ or $S$.
\item Intersection $R\cup S$  is a set of tuples that are in both $R$ and $S$. 
\item Difference $R-S$ is a set of tuples that are in $R$ but not in $S$.

\end{itemize}

Lets have relations $R$ and $S$. If we want to apply some set operation both relations must have the same set of attributes. If we want to compute set theoretic union, difference or intersections the oder of columns must be the same in both relations. 
We can also use renaming operations if relations doesn't have same number of attributes.

\subsubsection{Projection}
\label{projection}
$Projection$ operator $\pi$ produces from relation R new Relations with reduced set of attributes. Result of a expression $\pi_{A_1,A_3,A_4,...,A_N}(R)$ is relation $R$ with attributes $A_1,A_3,A_4,...,A_N$. 

\subsubsection{Selection}
If we apply operator selection $\sigma$ on Relation $R$ with condition $C$ we get a new relation with same attributes and tuples, which satisfy given condition. For example $\sigma_{A_1=4}(R)$.

\subsubsection{Cartesian product}
Cartesian product of two sets $R$ and $S$ creates a set of pairs by choosing the first element of pair to be any element from R and second element of pair to be any element of S. Cartesian product of relations similar. We pair tuples from $R$ with all tuples from $S$.

\subsubsection{Natural joins}
We usually don't want to pair all of the tuples from $R$ to all tuples from $S$. We can par tuple in some other way. The simples join is called natural join of $R$ and $S$ ($R \Join S$). Let schema of $R$ be $R(r_1,r_2,..,r_n,c_1,c_2,...c_n)$ and schema of $S$ be  $S(s_1,s_2,..,s_n,c_1,c_2,...c_n)$. In natural join we pair tuple $r$ from relation $R$ to tuple $s$ from relation $S$ only if $r$ and $s$ agree on all attributes with same name (in this case $c_1,c_2,...c_n$).

\subsubsection{Theta joins}
Natural join forces us to use one specific condition. In may cases we want to join relation with some other condition. For this purpose we have theta-join. The notation for joining relation $R$ and $S$ based on condition $C$ is $R\Join_C S$. The result is constructed in following way:

\begin{enumerate}
\item Make Cartesian product of $R$ and $S$
\item Use selection with condition $C$.
\end{enumerate}

Basically $R\Join_C S=\sigma_C(R \times S)$

\subsubsection{Renaming}

In order to control name of attributes or relation name we have renaming operator. We can use operator $\rho_{S(A_1,A_2,...,A_n)}(R)$. Result will have the same tuples as $R$ but relation will be called S and attributes will be renamed to $(A_1,A_2,...,A_n)$.

\subsection{Relational operations on bags}

Commercial database system almost never are based purely on bags. $Bag$ is a multi-set. Only operation that behave differently are intersection union and difference. 

\subsubsection{Union}
Bag union of $R \cup S$ we just add all tuples from $S$ and $R$ together. If tuple $t$ appears in $R$ $m$-times and in $S$ $n$-times then in  $R \cup S$ will $t$ appear $m+n$ time. Both $m$ and $n$ can be zero.

\subsubsection{Intersection}

Lets have tuple $t$ that appears in $R$ $m$-times and $S$ $n$-times. In the Bag intersection $R \cap S$ will be $t$ $min(m,n)$-times.

\subsubsection{Difference}
Every tuple $t$ that appears in $R$ $m$-times and $S$ $n$-times, will appear $max(0,m-n)$ times in bag $R-S$.


\subsection{Extended operators of Relational algebra}

We will introduce extended operators that proved useful in many query languages like SQL.


\subsubsection{Duplicate elimination}
This operator $\delta(R)$ returns set consisting of one copy of every tuple that appears in bag R one or more times.

\subsubsection{Aggregate operations}

Aggregate operators such as sum are not relational algebra operator but are used  by grouping operator. They apply on column and produce one number as result. The standard operators are $SUM$, $AVG$(average), $MIN$, $MAX$ and $COUNT$.


\subsubsection{Grouping operator}

We often doesn't want to compute aggregation function for entire column. We rather compute this function on for some group of columns. For example we can compute average salary for every person in database, or we can group them by companies and get every salary in every company. 

For this purpose we have grouping operator $\gamma_L(R)$. $L$ is a list of:

\begin{enumerate}
\item Attribute of $R$ by which $R$ will be grouped.
\item Aggregation operator applied on a attribute of relation.
\end{enumerate}

Relation computed by expression $\gamma_L(R)$ is constructed:

\begin{enumerate}
\item Relation will be partitioned into groups. Every group contains all tuples which have same value in all grouping attributes. If there is no grouping attributes, all tuples will be in one group.
\item For each group operator produces one tuple consisting of:
 \begin{enumerate}
 	\item Grouping attributes values for group.
 	\item Results of aggregations over all tuple of processed group.
 \end{enumerate}
\end{enumerate}

Duplicate elimination operator is a special case of grouping operator. We can express $\delta(R)$ with $\gamma_{L}(R)$, where $L$ is a list of all attributes of $R$.

\subsubsection{Extended projection operator}

We can extend classical projection operator $\pi_L(R)$ introduced in chapter \ref{projection}. We denote it also $\pi_L(R)$ but projection list can have following elements:

\begin{enumerate}
\item Attribute of R, which means attribute will appear in output.
\item Expression $x = y$, attribute $y$ will be renamed to $x$.
\item Expression $x = E$, where $E$ is an expression created from attributes from R, constants, arithmetic, string  and other operators. $x$ is new name. For example $x=e*(1-l)$.
\end{enumerate}



\subsubsection{The sorting operator}

In several situations we want the output of query to be sorted. Expression $\tau_L(R)$, where $R$ is relation, $L$ is list of attributes with additional information about sort order, is relation with same tuples like $R$ but different order of tuples. Example:  $\tau_{A_1:A,A_2:D}(R)$ will sort relation $R$ by attribute $A_1$ ascending and tuples with same $A_1$ value will be additionally sorted by their $A_2$ value descending. 

\subsubsection{Outer joins}
Lets have join $R\Join_C S$. We call tuple $t$ from relation $R$ or $S$ $dangling$ if
we didn't find any match in relation $S$ or $R$. Outer join $R\Join^\circ_C S$ is formed by creating $R\Join_C S$ and adding dangling tuples from $R$ and $S$. The added tuples must be filled with special $null$ value in all attributes they don't have but appear in join result.

Left/right outer join is outer join but we only add dangling tuples from left/right relation.

\section{Optimizations of relational algebra}

After initial logical query plan is generated, we can apply some heuristics to improve it, using some algebraic laws that hold for relational algebra. 


\subsection{Commutative and associative laws}
Commutative and associative operators are Cartesian product, natural join, union and intersection. Theta join is commutative but generally is not associative. But if the conditions makes sense where they where positioned, then theta join is associative.
That means we can make following changes to algebra tree:

\begin{itemize}
\item $R \oplus S = S \oplus R$

\item $(R \oplus S) \oplus T = R \oplus (S \oplus T)$
\end{itemize}
$\oplus$ stands for $\times$, $\cap$, $\cup$, $\Join$ or $\Join_C$.

\subsection{Laws involving selection}

Selection are very important for improving logical plan. They usually reduce size of relation markedly so that's why we need to move them down the tree as far as possible.
We can change order of selections:

\begin{itemize}
\item $\sigma_{C_1}(\sigma_C{_2}(R)) = \sigma_C{_2}(\sigma_C{_1}(R))$
\end{itemize}
Sometimes we cannot push whole condition but we can split it:

\begin{itemize}
\item $\sigma_{C_1~AND~C_2}(R)=\sigma_{C_1}(\sigma_{C_2}(R))$

\item $\sigma_{C_1~OR~C_2}(R)=\sigma_{C_1}(R) \cup_S \sigma_{C_2}(R)$
\end{itemize}
Last law works only when $R$ is a set. $\cup_S$ stands for set union. We can push selection down union, it has to be pushed to both branches:
\begin{itemize}
\item $\sigma_{C}(R \cup S)=\sigma_{C}(R) \cup \sigma_{C}(S)$
\end{itemize}

When pushing selection through difference we must push it to first branch. Pushing to second branch optional. Laws for difference: 
\begin{itemize}
\item $\sigma_{C}(R-S)=\sigma_{C}(R) - \sigma_{C}(S)$
\item $\sigma_{C}(R-S)=\sigma_{C}(R) - S$
\end{itemize}

Following laws allow to push selection down both arguments. Let's have selection $\sigma_C$. We can push it to the branch, which contains all attributes used in $C$. If $C$ contains only attributes of $R$:
\begin{itemize}
\item $\sigma_{C}(R \oplus S)=\sigma_{C}(R) \oplus S$
\end{itemize}
$\oplus$ stands for $\times$, $\cap$, $\cup$, $\Join$ or $\Join_C$. If relation $S$ and $R$ contains all attributes of $C$ we can also use following law:
\begin{itemize}
\item $\sigma_{C}(R \Join S)=\sigma_{C}(R) \Join  \sigma_{C}(S)$
\end{itemize}


\subsection{Laws involving projection}
Principle for manipulation with projections that we can add projection anywhere in the tree as long as it only eliminates attributes which are not used anymore and don't appear in query result.
\subsection{Laws involving joins and products}

We have more laws involving selection that follow directly from definition of the join:
\begin{itemize}
\item $\sigma_{C}(R \times S)=R \Join_{C} S$
\item $R \Join S=\pi_L(\sigma_{C}(R \times S))$, $C$ is condition that equates each pair of attributes of $R$ and $S$, which have the same name and $L$ is a list of attributes of relation $R$.
\end{itemize}



\section{Physical plan generation}

After we optimized logical plan, we need to create physical plan. We generate many physical plans a choose one with least estimated cost to run it. This approach is called cost-based enumeration.

For each physical plan we select 
\begin{enumerate}
\item An order of grouping and joins.
\item An algorithm for each operator. For example if we use join based on hashing or sorting.
\item Additional operators which are not presented in logical plan. For example we can sort relation in order to user faster algorithm which assumes that it's input is sorted.
\item The way in which arguments are pass to between operators. We can use iterators for it or store result on hard drive.
\end{enumerate}

\subsection{Size estimations}

The costs of evaluating physical plan are based of estimated size of interme\-dia\-te relations. Ideally we want out estimation to be accurate, easy to compute and logically consistent(size of relation doesn't depend on how relation is computed). We will present simple rules, which will give us give us good estimations in most situation. Goal of estimating sizes is not predict exact size of relation, even an inaccurate sizes will help us with plan generation.

In this section we will use following conventions:

\begin{itemize}
\item $T(R)$ is number of tuples in relation R.
\item $V(R,a)$ number of distinct values in attribute $a$. 
\item $V(R,[a_1,a_2,...,a_n])$ is number of tuples in $\delta(\pi_{a_1,a_2,...,a_n}(R))$
\end{itemize}

\subsubsection{Estimating the size of projection}

Projection is only operator which size of result is compatible. It doesn't change number ot tuples, their lengths change.

\subsubsection{Estimating the size of selection}

Selection reduces number of tuples.Lets have $S=\sigma_{A=c}(R)$, where $A$ is a attribute of $R$ and $c$ is a constant. Recommended estimation is:
\begin{itemize}
\item $T(S)=T(R)/V(R,A)$
\end{itemize}

More problematic estimation is when selection involves inequality comparison Lets have $S=\sigma_{A<c}(R)$. In average half the tuple satisfies condition, but usually queries select only a small fraction from all tuples. Therefore the estimation is:
\begin{itemize}
\item $T(S)=T(R)/3$
\end{itemize}

For selection where condition is in form $C_1~and~C_2~and~...~and~C_N$ we can treat selection as a cascade of simple selections and estimate size for every simpler condition

When selection involves $not$ and we have $S=\sigma_{not(C)}(R)$ we can use following estimation:
\begin{itemize}
\item $T(S)=T(R)-T(\sigma_C(R))$
\end{itemize}

Little more complicated is when condition involved an $or$ of conditions. Lets have expression $S=\sigma_{C_1~or~C_2}(R)$. We can assume that $C_1$ and $C_2$ are independent. Size of $S$ is:

\begin{itemize}
\item $T(S)=T(R)(1-(1-\dfrac{m_1}{T(R)})(1-\dfrac{m_2}{T(R)}))$
\end{itemize}

Expression $1-\dfrac{m_1}{T(R)}$ is fraction of tuples which doesn't satisfy condition $C_1$ and $1-\dfrac{m_2}{T(R)}$ is fraction of tuples which doesn't satisfy condition $C_2$. Product of these numbers are the fraction of tuples from $R$ which are not in result. One minus the product gives us fraction of tuples in $S$.

\subsubsection{Estimating the size of join}

a

\subsubsection{Estimating the size of union}

\subsubsection{Estimating the size of intersection}

\subsubsection{Estimating the size of difference}

\subsubsection{Estimating the size of grouping}


\subsection{Enumerating plans}

\subsection{Choosing join order}

\subsection{Choosing physical algorithms}

table
select
join

\chapter{Analysis}
\label{analysis}
Used data structures and algorithms in the implemented compiler are discussed in this chapter.
\section{Format of relational algebra}

In this section we present relation algebra operators which are the input of the compiler. Our relational algebra contains the following operators:
\begin{enumerate}
\item Projection -- we used extended projection $\pi_L$ which removes columns, computes new ones using expressions and renames attributes.

\item Table reading operator which is a leaf of the algebra tree. For this operator we need to provide the following arguments:
\begin{itemize}
\item table name.
\item information about indices (name, columns and sort order).
\item read columns.
\end{itemize}
\item Join - we used theta join $\Join_C$ operator where $C$ is condition having following format:
\begin{itemize}
\item Condition can be empty and in this case join represents Cartesian product.
\item $a_1=b_1~and~a_2=b_2~and~a_3=b_3~and...and~a_n=b_n$, where $a_k$ belongs to the first relation and $b_k$ belongs to the other relation.
\item $a_1\oplus b \ominus a_2$, where $a_1$ and $a_2$ belongs to one input and $b$ belongs to second input. Signs $\oplus$ and $\ominus$ mean $<$ or $\leq$.

\end{itemize}

In addition to condition, we need to specify output attributes of the join. These attributes can come from both inputs and we can optionally assign them a new name. Assigning new attribute name is useful when some of the attributes have the same name.

The other types of joins are not directly supported, but they can be replaced with the cross join followed by selection.
\item Anti join operator which was not presented with other algebra join operators. Output of the expression $R \ltimes_C S$ is relation with tuples from $R$, for which do not exist any tuple from $S$ that satisfies condition $C$. We can use join and anti join to express outer join.
 
The anti join can replace difference operator. The expression $R-S$ equals $R \ltimes_C S$, where $C$ is condition that equates each pair of attributes of $R$ and $S$ with the same name.
 
Including the anti join in our relation algebra eliminates the need of usage the outer join and the difference and our relational algebra is simpler.

Condition $C$ of anti join $R \ltimes_C S$ has the following format:
\begin{itemize}
\item $a_1=b_1~and~a_2=b_2~and~a_3=b_3~and...and~a_n=b_n$, where $a_k$ belongs to first the relation and $b_k$ belongs to the other relation.
\end{itemize}
In every anti join we need to specify its output attributes with optional new name. The anti join can output only columns from the first relation. 
\item Group operator $\gamma_L$, where L is non empty list of group attributes and aggregate functions. Supported aggregate functions are $min$, $max$, $sum$ and $count$. The function $avg$ is not supported but it can easily computed using $sum$ and $count$. All mentioned functions except $count$ take one input attribute and the function $count$ has empty input. 

As mentioned before, group operator is more general version of the duplicate elimination which is not included in our algebra.
\item Sort operator $\tau_L$, where $L$ is a non empty list of attributes with sort directions.
\item Bag union $\cup$. Both input relations has to have the same names and types of attributes. The set union can be computed using bag union and grouping operator for duplicates elimination.
\item Selection used in our algebra does not differ from selection from classical relational algebra.

\end{enumerate}

Designed relational algebra works with bags.

\section{Physical algorithms}

In this section we enumerate and describe compiler's output algorithms. We assume that execution environment has enough memory and physical operators do not have to store intermediate result on hard drive.

The following algorithms are generated by the compiler:
\begin{itemize}
\item \texttt{Filter} - this algorithm reads input tuples and outputs tuples satisfying given condition. Output does not have to be sorted the same way as input.
\item \texttt{Filter~keeping~order} - this algorithm reads input tuples and outputs tuples satisfying given condition. Output has to be sorted the same way as input.
\item \texttt{Hash~group} - operator groups tuples using hash table and for every group of tuples aggregate functions are computed.
\item \texttt{Sorted~group} - operator groups sorted tuples and computes aggregate functions. The input has to be sorted by group attributes.
\item \texttt{Column~operations} - this is an implementation of extended projection algebra operator. 
\item \texttt{Cross~join} - operator computes Cartesian product of two relations.
\item \texttt{Hash~join} - operator uses hash table to compute join of two relations $R$ and $S$ with condition $C$, where $C$ has the following format: $r_1=s_1~and~r_2=s_2~and~...~r_n=s_n$. Attributes $(r_1,r_2,...,r_n)$ belong to the relation $R$ and $(s_1,s_2,...,s_n)$ are from the relation $S$.
\item \texttt{Merge~equijoin} - algorithm takes advantage of sorted inputs to compute join with condition $C$, where $C$ has the same format like condition in \texttt{Hash~join}. 
\item \texttt{Merge~non~equijoin} - operator computes theta join with condition $a_1\oplus b \ominus a_2$, where $a_1$ and $a_2$ belong the first input and $b$ belongs to the second input. Signs $\oplus$ and $\ominus$ means $<$ or $\leq$. Input relations has to be sorted by the attributes in the join condition.
\item \texttt{Hash~anti~join} -  algorithm computes anti join with condition $C$  using hash table. Condition $C$ have the same format like condition in \texttt{Hash~join}
\item \texttt{Merge~anti~join} - operator takes advantage of sorted inputs to compute anti join with condition $C$, where $C$ has same format like condition in \texttt{Hash~join}. 
\item \texttt{Table~scan} - operator scans whole table from hard drive.
\item \texttt{Scan~and~sort~by~index} - operator scans whole table from hard drive using index. Output will be sorted by columns on used index.
\item \texttt{Index~Scan} - this algorithm uses index to read tuples from table satisfying given condition.
\item \texttt{Sort} - this algorithm sorts input. Input can be presorted and in this case, operator does only partial sorting.
\item \texttt{Union} - an implementation of bag union.

\end{itemize}


\section{Architecture}
The architecture of implemented tool is displayed in the figure~\ref{fig:compilerarchitecture}.

\begin{figure}[h!]
  \centering
    \includegraphics[width=0.5\textwidth]{compilerarchitecture}

      \caption{Compiler architecture.}
          \label{fig:compilerarchitecture}
\end{figure}

The relational algebra tree is read from XML file. For this format we decided for the following reasons:
\begin{itemize}
\item XML has tree--like structure.
\item For validation we only need to write schema.
\item There are already implemented tools for parsing.
\item There is no need to write input parser.
\end{itemize}

The relational algebra tree is checked in the \texttt{Semantic analyzer}. This component checks if all of the used attributes are in the input relation. \texttt{Semantic analyzer} searches for duplicate named attributes and reports them as an error. 

Semantically correct tree is processed by component \texttt{Algebra grouper} that groups neighboring joins into one. Thanks to this operation we can later choose the fastest way to join multiple relations.

Algebra tree with grouped joins is optimized. \texttt{Algebra optimizer} pushes the selections down the tree. This component also applies following operation:
\begin{itemize}
\item  $\sigma_C(R\Join_D S)= R\Join_{D~and~C} S$, where $C$ has the following format: $r=s$, $r$ belong to $R$ and $s$ belongs to $S$.
\end{itemize}

Optimized algebra tree is processed by \texttt{Algebra compiler}, which generates physical plan. This physical plan is not yet final. Its sort operator's parameters can represent multiple ways of sorting relation. When sorting relation before grouping, we have multiple possibilities how to sort current relation and we can decide later which way is the best.

Final plan is an output of the component named \texttt{Sort resolver}. This component resolved unknown sort order in sort operators and produces final plan, which is converted to Bobolang language.

Implemented tool  will be the back end of the compiler and it does not check column. We assume that the front end parsing the text will handle types. Types of columns are only copied to the output and we assume that column types do not contain any errors.

\section{Data structures}

Data structures used in implemented tool are presented in this section.

Relational algebra is stored in the polymorphic tree. Every node stores its parameters, pointer on the parent in the tree and pointers on its children node. No other structure was considered for this representation since this is efficient way to store logical plan. It allows to easily add and remove relational algebra operators.
The example of this representation can be found in the Figure~\ref{fig:groupalgebra}. It is representing simple query reading whole table. Read relation is grouped and aggregation functions are computed. The result is sorted at the end. Leaf of the tree stores the following information:
\begin{itemize}
\item List of indices on this table.
\item List of columns with their names, types and number of unique values
\item Size of the relation.
\end{itemize}
\begin{figure}[h!]
  \centering
    \includegraphics[width=0.8\textwidth]{groupalgebra}

      \caption{Example of relational algebra structure.}
          \label{fig:groupalgebra}
\end{figure}

We chose the same structure for physical plan. The advantage of storing physical plan into polymorphic tree is to ability to easily add new root node. The example of this representation can be found in the Figure~\ref{fig:groupplan}. This Figure contains one of the possible physical plans for relational algebra shown in the Figure~\ref{fig:groupalgebra}. For reading table we used algorithm \texttt{Table scan}, then we hashed input by requested columns. Result is sorted in \texttt{Sort} operator. Every nodes stores additional informations like output attributes, estimated run time and size of output relation.

\begin{figure}[h!]
  \centering
    \includegraphics[width=0.6\textwidth]{groupplan}

      \caption{Example of physical plans structure.}
          \label{fig:groupplan}
\end{figure}

Physical and logical plans also contain expressions. The expressions are stored in polymorphic expression tree. Example of this structure can be found in the~Figure~\ref*{fig:expressiontree}.
\begin{figure}[h!]
  \centering
    \includegraphics[width=0.6\textwidth]{expressiontree}

      \caption{Example expression tree representing expression  $X*Y+874$.}
          \label{fig:expressiontree}
\end{figure}

More complicated structure was used for storing sort parameters. This structure is stored in every physical sort operator to determine how the relation can be sorted. 

If we want to use sort based group operator and it groups by two columns, we have multiple possibilities for sorting relation. To use sort based algorithm for evaluating expression $\gamma_{x,y}(R)$ we can sort relation in four possible ways:
\begin{itemize}
\item $x:A,y:A$
\item $x:A,y:D$
\item $x:D,y:A$
\item $x:D,y:D$
\end{itemize}
$A$ means ascending and $D$ is abbreviation for descending.

There are multiple possibilities of sorting relation $R$ before applying  $R\Join_{r_1=s_1~and~r_2=s_2} S$. Relation $R$ can be sorted the following ways:
\begin{itemize}
\item $r_1,r_2$
\item $r_2,r_1$
\end{itemize}
Order, how to sort columns is arbitrary.

We also want to store information about equality of sort column. After applying merge join $R\Join_{r_1=s_1} S$, the result can sorted by $r_1$ or $s_1$. All this requirements were use to design structure for store sort parameters without enumerating all possible sort orders.

\begin{figure}[h!]
  \centering
    \includegraphics[width=0.9\textwidth]{sortparameters}

      \caption{Structure storing parameters for sort.}
          \label{fig:sortparameters}
\end{figure}

In figure \ref{fig:sortparameters} we display an example of sort parameters, which sorts by 6 columns. It contains one or more columns group. The order of columns groups cannot be changed. Order of columns in groups is arbitrary. It means that $F$ has to be on sixth place, but column $E$ can be on forth of fifth place. Every column contain information about sort order: $ASC$ (ascending), $Desc$ (descending) or $UNK$ (unknown - can be ascending or descending). We store list of equal attributes for every column. In case a projection operator removes columns $A$ from relation we can replace removed with attributes $X$, $Y$ or $Z$ in sort parameters.
The Figure~\ref{fig:sortparameters} represents many sort order possibilities, we enumerate only some of them:
\begin{enumerate}
\item $A:ASC,C:DESC,B:ASC,H:DESC,D:ASC,F:DESC$
\item $C:DESC,B:ASC,Z:ASC,H:DESC,D:DESC,F:DESC$
\item $B:ASC,C:DESC,A:ASC,E:ASC,D:DESC,F:DESC$
\item $C:DESC,B:ASC,Y:ASC,D:ASC,H:ASC,F:DESC$
\end{enumerate}



\section{Optimization}

In this section we describe algebra optimization, which was implemented to improve logical plan.

Before we start with optimizations we need to prepare logical plan. We group joins algebra nodes and expressions connected with $and$ and $or$. We work down the algebra tree. If we find join we convert to it grouped join. If one of its children is join, then we merge it with grouped join. This representation is used for choosing faster order join. Conditions are grouped the same way. From expression tree $a=2~and~(b=2~and~c=2)$ we create $AND(a=2,b=2,c=2)$. This representation is useful for splitting condition into simpler conditions.

We implemented very important optimization: pushing selection down the tree. Every selection is spitted into selections with simpler conditions. Every selection is moved down the tree as much as possible. In this phase we used the following rules ($\sigma_C$ is being pushed down):
\begin{enumerate}
\item $\sigma_C(\sigma_D(R))=\sigma_D(\sigma_C(R))$
\item $\sigma_C(\pi_L(R))=\pi_L(\sigma_C(R))$, it works only if $C$ does not contain new computed columns in extended projection. We also need to rename columns in condition $C$ in case projection renamed some of the condition columns.
\item  $\sigma_C(R \Join_D S)$ can be rewritten as
\begin{enumerate}
\item $\sigma_C(R) \Join_D S$ if $C$ contains only columns from $R$.
\item $R \Join_D \sigma_C(S)$ if $C$ contains only columns from $R$.
\item $R \Join_{D~and~C} S$ if $C$ is in form $a=b$ where $a$ belong the relation $R$ and $b$ comes from the relation $S$.
\end{enumerate}
\item $\sigma_C(R \ltimes_C S)=\sigma_C(R)\ltimes_C S$
\item $\sigma_C(R\cup S)=\sigma_C(R)\cup \sigma_C(S)$
\end{enumerate}

\section{Generating physical plan}

We try to choose easiest method for generating plans. The decision was between heuristic method and dynamic programming. It would be probably the same amount of source code. We chose dynamic programming, because it can give better results. Using this method, we generate all possible plans for each node and choose the fastest plan.

We process logical plan from leafs. For every leaf we generate all possible physical algorithms and we insert resulting plans into heap, where we keep $c$ fastest plans for current node. Variable $c$ is a constant set in compiler.
For every algebra node we used plans generated in its children to generate new plan. This way we continue up the algebra tree to the root.

Physical plans are compared based on estimated run time. Every operator stores its estimated run time. The sum of run times of all the operators is estimated run time of the whole physical plan.
Very important are equations, which compute estimated time for physical algorithm. They depend on size of input relation. Modifying them can resolve in getting better physical plans. For example, if physical algorithm \texttt{Hash join} takes too much time because it accesses random parts of memory, we can modify estimated times so sort with merge join will be preferred.

Crucial are informations about size of tables. If they are not provided in the input we use default value and physical plan will be probably worse. Other important parameter is number of unique values attribute. Size of join depends on it and since joins usually takes significant amount of time, it is important to have as precise values as possible. 

\subsection{Join order selecting algorithm}


In section \ref{joinOrder} we presented algorithm choosing join order. We should choose order of joins and then assign join algorithms. These operations are done in one phase for the following reason: In case we do not have information about table sizes, we cannot determine join order because all orders have the same estimated run time. In this situation we can start by joining relations which are sorted to get a faster plan.

We are using two algorithms dynamic programming and greedy algorithm. Version of used dynamic programming algorithm is enumerating all possible trees. This algorithm can output us very good plans but has exponential complexity. We use it only if number of joined relations in grouped join node is small. For joining more relations we use faster greedy algorithm which generated only left or right deep trees. Time complexity of greedy algorithm is not exponential but only polynomial.

Input in both algorithm is set of plans for every input join relation.

\subsubsection{Dynamic programing for selection join order}

We use a variation of algorithm described in section \ref{dymanicalgorithm}. Input relations are numbered from $1$ to $n$. We used table for storing plans, which key is non empty subset of the set ${1..n}$.

We only store $k$ best plans in every table cell, where $k$ is constant set in the compiler. It represents best plans that were created by joining inputs stored in the key of table entry.

We begin by storing input plans into table entry identified by the set containing number of input relation. In first iteration we fill tables entries, which key has two values, by combining plans from entries with key size $1$.

We continue by computing plans for entries, with key size $3,4...n$. Key of the current entry is spited in all possible pairs of non empty disjunctive subsets. We take plans from table entries identified with subsets and we generate new plans combining them. We only store $k$ fastest plans in the current table cell.
We continue until we compute plans for table entry identified by key ${1..n}$ These plans are our result.

Time complexity is at least exponential since we generate all subsets of $n$ relations, this value is $2^n$.


\subsubsection{Greedy algorithm for selection join order}
 This is a variation of algorithm described in section \ref{greedyalgorithm}. We begin by creating joins for every pair of relations. From created pairs we choose the fastest $k$ trees.
 
 In every iteration we generate new trees by adding new relation to every tree. Only $k$ best trees are chosen to continue to the following iteration. We iterate until we create join tree containing every input relation.
 
 Time complexity is $O(n^2)$. In every iteration we generate from every tree maximal $n$ new trees, but we keep only $k$ of them for next iteration. Number of iterations is $n-1$, because all trees grow by one in every iteration. Number $k$ is a constant so it does not change time complexity.
 
\subsection{Resolving sort parameters}

After we generated physical plan we need to decide what sort parameters in sort operator to use. We work down the tree and store information about sort order of the relation. Based on that, we adjust sort parameters or just choose the arbitrary sort order if possible.


\chapter{Implementation}

In this chapter we describe implementation details in developed software and describe it's functionality on examples. We will use following example to show optimizations and compiling:


\begin{verbatim}
select
    l_orderkey,
    sum(l_extendedprice*(1-l_discount)) as revenue,
    o_orderdate,
    o_shippriority
from
    customer,
    orders,
    lineitem
where
    c_mktsegment = '[SEGMENT]'
    and c_custkey = o_custkey
    and l_orderkey = o_orderkey
    and o_orderdate < date '[DATE]'
    and l_shipdate > date '[DATE]'
group by
    l_orderkey,
    o_orderdate,
    o_shippriority
order by
    revenue desc,
    o_orderdate;
\end{verbatim}

This example is taken from Tpc benchmark \cite{benchmark}. $[DATE]$ and $[SEGMENT]$ are constants. In this benchmark there are no indexes on tables. Columns which starts wit o\_ are from table order, columns which beginning with l\- are from table lineitem and columns starting with \_c belongs table customers.



\section{Input}

As mentioned input is XML containing logical query plan. In this section we describe it's structure. 

\subsection{Sort}

On root of every tree is sort, even if output hasn't been sorted. in this case it has empty parameters. This is an example of sort in algebra tree:


\lstset{
  language=XML,
  morekeywords={encoding,
    xs:schema,xs:element,xs:complexType,xs:sequence,xs:attribute}
}
\begin{lstlisting}
<?xml version="1.0" encoding="utf-8"?>
<sort xmlns:xsi="http://www.w3.org/2001/XMLSchema-instance"
 xsi:noNamespaceSchemaLocation="algebra.xsd">
  <parameters>
    <parameter column="revenue" direction="desc" />
    <parameter column="o_orderdate" direction="asc" />
  </parameters>
  <input>
  ...
  <input>
</sort>
\end{lstlisting}

Sort is a root element of XML file. Inside parameters is specified how to sort relation. In this example we have sort $\tau_{revenue:desc,o\_orderdate:asc}(...)$. In element input there should be one other algebra tree node.

\subsection{Group}

Next example display group node:

\begin{lstlisting}
<group>
  <parameters>
    <group_by column="l_orderkey"/>
    <group_by column="o_orderdate" />
    <group_by column="o_shippriority"/>
    <sum argument="x" output="revenue"/>
  </parameters>
  <input>
  ...
  <input>
</group>
\end{lstlisting}

This node represents expression $\gamma_{l\_orderkey,o\_orderdate,o\_shippriority,x=sum(x)}(...)$. Group element has to have at least one group by parameter or at least one aggregate function. Inside element input there should be one other operator.

\subsection{Selection}

This is an example of selection:

\begin{lstlisting}

<selection>
  <parameters>
    <condition>
      <lower>
        <constant type="date" value="today"/>
        <column name="l_shipdate"/>
      </lower>
    </condition>
  </parameters>
  <input>
  ...
  </input>
</selection>
\end{lstlisting}
This example represent following expression: $\sigma_{today<l_shipdate}$. In condition element we can have multiple conditions connected by $and$ or $or$ elements. Input algebra supports operators $=$,$<$ and $\leq$. In the leafs of expression tree there can be only column or constant element. We also can call a boolean function from condition, which is represented in following example.

\begin{lstlisting}
<condition>
  <boolean_predicate  name="like">
    <argument>
      <column name="x"/>
    </argument>
    <argument>
      <constant type="int" value="445" />
    </argument>
  </boolean_predicate>
</condition>
\end{lstlisting}

While using boolean predicate it has to be supported by runtime(Bobox operators). Compile doesn't check it's existence.

\subsection{Join}

Join without condition is considered to be cross join. We can use join with multiple equal conditions or with simple unequal condition. First example contains equal conditions:

\begin{lstlisting}
<join>
  <parameters>
    <equal_condition>
      <equals>
        <column name="a"/>
        <column name="b"/>
      </equals>
      <equals>
        <column name="c"/>
        <column name="d"/>
      </equals>
    </equal_condition>
    <column name="a" input="first"/>
    <column name="b" input="second"/>
    <column name="c" input="first"/>
    <column name="d" input="second" newName="e" />
  </parameters>
<input>
...
</input>
\end{lstlisting}

This example represents join with condition $a=b~and~c=d$. In join equal conditions has to be first column from first relation and second column from second relation. In example $a$ and $c$ are from first input and $b$ and $d$ are from the other one. Joins doesn't copy to output all column from both input relations. After condition we have to specify non empty sequence of columns. In every column we specify it's name and number of input. We can also rename join output column by using attribute $newName$. In example we renamed column $d$ to $e$.

Next example shows also join but with inequality condition:

\begin{lstlisting}
<join>
  <parameters>
    <less_condition>
      <and>
        <lower_or_equals>
          <column name="a1"/>
          <column name="b"/>
        </lower_or_equals>
      <lower_or_equals>
        <column name="b"/>
        <column name="a2"/>
      </lower_or_equals>
    </and>
  </less_condition>
  <column name="a1" input="first"/>
  <column name="b" input="second"/>
  <column name="a2" input="first"/>
</parameters>
<input>
...
</input>
</join>
\end{lstlisting}
This example represents join with condition $a1\leq b\leq a2$. In first sub condition first column has to be from first input, but in second sub condition first column has to be from second input. Also instead $lower\_or\_equals$ we can use just $lower$ condition. Rules for output column are same like in join with equal conditions.

In element $input$ of join there has to be two operators.

\subsection{Anti join}

\begin{lstlisting}
<antijoin>
  <parameters>
    <equal_condition>
      <equals>
        <column name="d"/>
        <column name="b"/>
      </equals>
    </equal_condition>
    <column name="d"/>
  </parameters>
<input>
...
<input>
</antijoin>
\end{lstlisting}
This is an example of antijoin with simple condition $d=b$. Structure is the almost same like join. Output columns can be only from first relation and we can also rename this columns.

\subsection{Table}
This is a leafs of algebra tree. It specifies name of read table, its columns and indexes. We can specify number of rows in the table to get better plans. If it is not specified we will assume that table has 1000 tuples. For every column we have to specify name and it's type. Other optional parameter is $number\_of\_unique\_values$. This number is important for estimating size of join. If it is not given, we will assume, that $number\_of\_unique\_values$ is size of table to power of $\frac{4}{5}$. This assumption is only experimental, since number of unique values can be from $0$ to size of table. Index can be clustered or unclustered. Table can have only one clustered index. In every index we specify on what attribute it is created. Here is an example or table algebra node:
 
\begin{lstlisting}
<table name="orders" numberOfRows="1500000">
  <column name="o_orderdate" type="int"/>
  <column name="o_shippriority" 
  type="int" number_of_unique_values="30000"/>
  <column name="o_orderkey" type="int"/>
  <column name="o_custkey" type="int" />
  <index type="clustered" name="index">
    <column name="o_orderdate" order="asc" />
    <column name="o_shippriority" order="asc" />
  </index>
</table>
\end{lstlisting}

\subsection{Union}
Union doesn't have any parameters, but columns from both input have tt have the same name. Here is an example:
\begin{lstlisting}
<union>
  <input>
  ...
  </input>
</union>
\end{lstlisting}

\subsection{Extended projection}
Following example of extended projection represents expression \\ $\pi_{l\_orderkey,o\_orderdate,o\_shippriority,x=l\_extendedprice*(1-l\_discount)}(...)$.
\begin{lstlisting}
<column_operations>
  <parameters>
    <column name="l_orderkey"></column>
    <column name="o_orderdate"></column>
    <column name="o_shippriority"></column>
    <column name="x">
      <equals>
        <times>
          <column name="l_extendedprice"/>
          <minus>
            <constant type="double" value="1"/>
            <column name="l_discount"/>
          </minus>
        </times>
      </equals>
    </column>
  </parameters>
  <input>
  ...
  </input>
</column_operations>
\end{lstlisting}

Extended projection contains list of columns. If columns is new computed values it contains elements representing expression tree. It cal also contain function call, which has to be supported by Bobox operators. Following example displays function call:


\begin{lstlisting}
<column_operations>
  <parameters>
    <column>
      <equals>
        <aritmetic_function name ="sqrt">
          <argument>
            <constant type="double" value="2"/>
          </argument>
        </aritmetic_function>
      </equals>
    </column>
  </parameters>
  <input>
    ...
  </input>
</column_operations>
\end{lstlisting}
 

\section{Output}

In this section we describe text output generated by implemented compiler.

\subsection{Filters}
Example: 
\begin{lstlisting}
Filter(double,double,int)->(double,double,int)
f(condition="OP_LOWER(OP_double_CONSTANT(4.8),1)"); 
\end{lstlisting}

Input and output columns are the same and they are numbered from 0.
This operator takes input of two double streams and integer stream and it filters by condition $4.8<(column~number~1)$. We have also another version of this operator, which guaranteers that input and output are sorted the same way. To use it we write $FilterKeepingOrder$ instead of $Filter$ in operator declaration. 

\subsection{Group}
Example: 
\begin{lstlisting}
HashGroup(string,string,int)->(string,int,int)
g(groupBy="1",functions="count(),max(2)");
\end{lstlisting}
Input columns are numbered from 0. Output columns consists from grouped columns and computed aggregate functions in the same order as in parameters. 
This example groups by column number 1 and computes aggregate function $COUNT$ and $MAX$. $MAX$ has as parameter column number 2. 

We have also sorted version of this operators. It assumes that input is sorted by group columns. To use it we write $SortedGroup$ instead of $HashGroup$ in declaration.

\subsection{Column operations}
Example: 
\begin{lstlisting} 
ColumnsOperations(int,int,int,int,int)->(int,int,int,double)
c(out="0,3,4,OP_TIMES(2,OP_MINUS(OP_double_CONSTANT(1),2))"); 
\end{lstlisting}
Input columns are numbered from 0. Output is specified in parameter $out$. It if contains number operator, copies input to output, otherwise it computes new column. 
This example copies columns number $0,3,4$ to output and computes new column with expression: $2*(1-column~number~2)$.

\subsection{Cross join}
Example:
\begin{lstlisting} 
CrossJoin(string,int)(int,string)->(string,string)
c(left="0,1",right="2,3",out="0,3");
\end{lstlisting}
Numbering columns from first input is specified in $left$ parameter and numbering columns from second input is specified in $right$ parameter. Join outputs only columns given in $out$ argument. 

\subsection{Hash join}
Example:
\begin{lstlisting}
HashJoin(int,int)(int,int,int,int)->(int,int,int,int,int,int)
h(left="0,1",right="2,3,4,5",out="0,1,2,3,4,5",
leftPartOfCondition="0,1","rightPartOfCondition="5,2"); 
\end{lstlisting}
Numbering columns from first input is specified in $left$ parameter and numbering columns from second input is specified in $right$ parameter. Join outputs only columns given in $out$ argument. This operators works only with equal condition, which is given in parameters $leftPartOfCondition$ and $rightPartOfCondition$. This example computes join with condition:\\
 $(column~0=column~5)~and~(column~1=column~2)$.

\subsection{Merge equijoin}
Example:
\begin{lstlisting}
MergeEquiJoin(int)(int)->(int,int))
m(left="0",right="1",out="0,1",leftPartOfCondition="0:D",
rightPartOfCondition="1:D");
\end{lstlisting}
Numbering columns from first input is specified in $left$ parameter and numbering columns from second input is specified in $right$ parameter. Join outputs only columns given in $out$ argument. Condition is given in parameters $leftPartOfCondition$ and $rightPartOfCondition$ and they also contain information how are inputs sorted. This example computes join with condition $(0==1)$. First input is sorted by column number 0 descending and the second input is sorted by column 1 descending.

\subsection{Merge non equijoin}
Example:
\begin{lstlisting}
MergeNonEquiJoin(date,date)(date)->(date,date,date)
m(left="0,1",right="2",out="0,1,2",
leftInputSortedBy = "0:A,1:A",rightInputSortedBy = "2:A",
condition="OP_AND(OP_LOWER_OR_EQUAL(0,2)
,OP_LOWER_OR_EQUAL(2,1))");
\end{lstlisting}

This operator joins sorted relations. Numbering from left(first) and right(second) input is specified in parameters $left$ and $right$. Parameters $leftInputSortedBy$ and $rightInputSortedBy$ store information about how are input relations sorted. Join condition is in parameter $condition$. Operator in this example joins by condition $column~0 \leq column~2\leq column~1$. First input is sorted by column $0$ ascending and column $1$ ascending and second input is sorted by column $2$ ascending.

\subsection{Hash anti join}
Example:
\begin{lstlisting}
HashAntiJoin(int)(int)->(int)
h(left="0",right="1",out="0",leftPartOfCondition="0",
rightPartOfCondition="1"); 
\end{lstlisting}

Column number from first input is specified in $left$ parameter and columns numbers from second input is specified in $right$ parameter. Join outputs only columns given in $out$ argument. Parameter $out$ can only contains columns from first input. And specifies columns, which goes to output. Condition is given in parameters $leftPartOfCondition$ and $rightPartOfCondition$.This example computes antijoin with condition $(column~0==column~1)$.

\subsection{Merge anti join}
Example:
\begin{lstlisting}
$MergeAntiJoin(int)(int)->(int)
$m(left="0",right="1",out="0",leftPartOfCondition="0:D",
rightPartOfCondition="1:D");
\end{lstlisting}

Numbering columns from first input is specified in $left$ parameter and numbering columns from second input is specified in $right$ parameter. Join outputs only columns given in $out$ argument. Operator copies to output only rows from first input for which doesn't exist row in second input satisfying given condition.
Condition is given in parameters $leftPartOfCondition$ and $rightPartOfCondition$ and they also contain information how are inputs sorted. This example computes join with condition $(column~0==column~1)$. First input is sorted by column number 0 descending and the second input is sorted by 1 descending.

\subsection{Table scan}
Example:
\begin{lstlisting}
TableScan()->(int,int,int,int)
t(name="lineitem",
columns="l_orderkey,l_shipdate,l_extendedprice,l_discount");
\end{lstlisting}
This operator scans table specified in parameter $name$ and reads only columns given in parameter $columns$.

\subsection{Scan And Sort By Index}
Example:
\begin{lstlisting}
ScanAndSortByIndexScan()->(string,string,int)
s(name="people",index="index",
columns="user_name,country,parameter"); 
\end{lstlisting}
Operator reads whole table given in $name$ using $index$ and reads columns specified in attribute $columns$.

\subsection{Index Scan}

Example:
\begin{lstlisting}
IndexScan()->(int,int)
i(name="customer",index="index2",columns="c_custkey,c_mktsegment",
condition="OP_EQUALS(1,OP_string_CONSTANT(SEGMENT))");
\end{lstlisting}
Operator reads part of table given in $name$ using $index$ and reads columns specified in attribute $columns$. Operator reads only rows satisfying condition given in attribute $condtion$.



\subsection{Sort}
Example:
\begin{lstlisting}
SortOperator(int,int)->(int,int)
s(sortedBy="0",sortBy="1:D");
\end{lstlisting}
Input and output columns are the same and they are numbered from 0. Parameter $sortedBy$ specifies by which columns is table sorted and parameter $sortBy$ specifies by which columns should table be sorted. Example is already sorted by $colum~0$ and will be sorted by $column~1$ descending.


\subsection{Union}
Example:
\begin{lstlisting}
Union(int,string)(string,int)->(int,string)
u(left="0,1",right="1,0",out="0,1");
\end{lstlisting}
Numbering columns from the first input is given in the $left$ parameter, from second input is given in the $right$ parameter and from the output is given in the $out$ parameter.




% Ukázka použití některých konstrukcí LateXu (odkomentujte, chcete-li)
% \include{example}

\chapter{Conclusion}


The aim of this thesis was to implement part of the SQL compiler. Created program reads input relational algebra, which is optimized. We implemented very effective logical plan optimization: pushing selections down the tree. Possible physical plans we enumerated using Selinger--Style Optimization method. In this phase we replaces algebra operators with physical plan. For estimated order of joins we implemented two different algorithms. Asymptoticly slower algorithm based on dynamic programming is used for estimated order of joins on smaller amount of relations. For larger amount of joined relations we provide faster greedy algorithm, which can generates less optimal join tree. While choosing order of joins we assigned physical algorithms. Merging assignment of physical algorithms and chose of join order can result in faster physical plan, in case we do not have informations about sizes of input relations. Physical plan is written to output in Bobolang language. Implemented compiler provides possibility to write algebra tree and physical plan to language Dot for debugging purposes.

After the introduction we described Bobox architecture and Bobolang language. Next chapter contains theory used to implement query transformer. Chapter analysis deals with description of used algorithms and important data structures used in implemented tool. Final chapter presents some implementation details of created program.

Created software is a first part of planed SQL compiler. Front end, which transforms text query to relational algebra, is not yet implemented. At the time of submitting this thesis, compiler was succesfully connected to Bobox, but not all of physical operators are implemented. We couldn't evaluate any queries to prove that generated plans are correct.

We tested software by transforming some simple queries and queries from benchmark\cite{benchmark} to physical plans. We can check generated plans by looking generated debug outputs. From this results we can say that generated plans are correct and also optimal. and this thesis fulfilled it's aim. 


Implemented tool can be improved by adding more logical plan optimizations.
After we run queries and measure their run time we can improve compiler time estimations used for selection of physical algorithms. 
We can add also support for more physical algorithms like nested loop joins.



%%% Seznam použité literatury
%%% Seznam použité literatury je zpracován podle platných standardů. Povinnou citační
%%% normou pro diplomovou práci je ISO 690. Jména časopisů lze uvádět zkráceně, ale jen
%%% v kodifikované podobě. Všechny použité zdroje a prameny musí být řádně citovány.

\def\bibname{Bibliography}
\begin{thebibliography}{99}
\addcontentsline{toc}{chapter}{\bibname}

\bibitem{bobox}
  D. Bednárek, J. Dokulil, J. Yaghob, and F. Zavoral.
  \emph{Bobox: Parallelization
  framework for data processing. Advances in Information Technology and
  Applied Computing}, 2012.
  
\bibitem{bobolang}
   Z. Falt, , D. Bednárek, K. Martin, J. Yaghob, and F. Zavoral.
  \emph{Bobolang - a language for parallel streaming applications.}  In 23rd international sym-
      posium on High-Performance Parallel and Distributed Computing. ACM,
      2014.
  
\bibitem{database}
   H. Garcia-Molina, J. D. Ullman, J. Widom.
  \emph{Database Systems The Complete Book}. Prentice Hall, 2002,
 ISBN 0-13-031995-3.

\bibitem{faltthesis}
  Zbyněk Falt.
  \emph{Parallel Processing of Data - Doctoral thesis}. Prague, 2013.

\bibitem{benchmark}
 \emph{ TPC BENCHMARK TM H},
 Standard Specification, Revision 2.15.0

\bibitem{xerces}
 \emph{ Xerces-C++},
http://xerces.apache.org/xerces-c/

\bibitem{doxygen}
 \emph{ Doxygen},
www.doxygen.org/

\end{thebibliography}


%%% Tabulky v diplomové práci, existují-li.
\chapwithtoc{List of figures}


%%% Přílohy k diplomové práci, existují-li (různé dodatky jako výpisy programů,
%%% diagramy apod.). Každá příloha musí být alespoň jednou odkazována z vlastního
%%% textu práce. Přílohy se číslují.
\chapwithtoc{Attachments}

\openright
\end{document}
