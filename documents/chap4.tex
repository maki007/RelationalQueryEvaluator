\chapter{Implementation}

In this chapter we describe implementation details in developed software and describe it's functionality on examples. We will use following example to show optimizations and compiling:


\begin{verbatim}
select
    l_orderkey,
    sum(l_extendedprice*(1-l_discount)) as revenue,
    o_orderdate,
    o_shippriority
from
    customer,
    orders,
    lineitem
where
    c_mktsegment = '[SEGMENT]'
    and c_custkey = o_custkey
    and l_orderkey = o_orderkey
    and o_orderdate < date '[DATE]'
    and l_shipdate > date '[DATE]'
group by
    l_orderkey,
    o_orderdate,
    o_shippriority
order by
    revenue desc,
    o_orderdate;
\end{verbatim}

This example is taken from Tpc benchmark \cite{benchmark}. $[DATE]$ and $[SEGMENT]$ are constants. In this benchmark there are no indexes on tables.

\section{Input}

As mentioned input is XML containing logical query plan. In this section we describe it's structure. 

\subsection{Sort}

On root of every tree is sort, even if output hasn't been sorted. in this case it has empty parameters. This is an example of sort in algebra tree:


\lstset{
  language=XML,
  morekeywords={encoding,
    xs:schema,xs:element,xs:complexType,xs:sequence,xs:attribute}
}
\begin{lstlisting}
<?xml version="1.0" encoding="utf-8"?>
<sort xmlns:xsi="http://www.w3.org/2001/XMLSchema-instance"
 xsi:noNamespaceSchemaLocation="algebra.xsd">
  <parameters>
    <parameter column="revenue" direction="desc" />
    <parameter column="o_orderdate" direction="asc" />
  </parameters>
  <input>
  ...
  <input>
</sort>
\end{lstlisting}

Sort is a root element of XML file. Inside parameters is specified how to sort relation. In this example we have sort $\tau_{revenue:desc,o\_orderdate:asc}(...)$. In element input there should be one other algebra tree node.

\subsection{Group}

Next example display group node:

\begin{lstlisting}
<group>
  <parameters>
    <group_by column="l_orderkey"/>
    <group_by column="o_orderdate" />
    <group_by column="o_shippriority"/>
    <sum argument="x" output="revenue"/>
  </parameters>
  <input>
  ...
  <input>
</group>
\end{lstlisting}

This node represents expression $\gamma_{l\_orderkey,o\_orderdate,o\_shippriority,x=sum(x)}(...)$. Group element has to have at least one group by parameter or at least one aggregate function. Inside element input there should be one other operator.

\subsection{Selection}

\subsection{Join}

\subsection{Anti join}

\subsection{Table}

\subsection{union}

\subsection{Extended projection}




 

\section{Output}

